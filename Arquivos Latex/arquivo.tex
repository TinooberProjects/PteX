
    
    %%%%%%  CABECALHO %%%%%%%%%%%%%%%%%%%%%%%%%%%%%%



\documentclass[
% -- opções da classe memoir --
12 pt,				% tamanho da fonte
openright,			% capítulos começam em pág ímpar (insere página vazia caso preciso)
%twoside,			% para impressão em verso e anverso. Oposto a oneside
oneside,
a4paper,			% tamanho do papel. 
% -- opções da classe abntex2 --
chapter=TITLE,		% títulos de capítulos convertidos em letras maiúsculas
%	section=TITLE,		% títulos de seções convertidos em letras maiúsculas
%	subsection=TITLE,	% títulos de subseções convertidos em letras maiúsculas
%	subsubsection=TITLE,% títulos de subsubseções convertidos em letras maiúsculas
% -- opções do pacote babel --
english,			% idioma adicional para hifenização
brazil	               % o último idioma é o principal do documento
]{abntex2}

%%%%%%%%%%%%%%%%%%%%%%%%%%%%%%%%%%%%%%%%%%%%%%%%%%%%%

\usepackage{ptex}


%%%%%%%%%%%%%%%%%%%%%%%%%%%%%%%%% CONFIGURAÇÃO DOS DADOS  DO TRABALHO
     \titulo{}
\autor{}
\local{}
\data{}
\orientador{}
\aprovadoem{}
\instituicao{}
\curso{}

\palavrachaveum{Palavra chave1 }
\palavrachavedois{Palavra chave2 }
\palavrachavetres{Palavra chave3 }
\tipotrabalho{Monografia}


\nomebancaum{TEste}
\titulobancaum{Mestre em Educação}
\bancauminstituicao{Instituto Federal de Ciência e e e ee}


\nomebancadois{TEste}
\titulobancadois{Mestre em Educação}
\bancadoisinstituicao{Instituto Federal de Ciência e e e ee}


\nomebancatres{TEste}
\titulobancatres{Mestre em Educação}
\bancatresinstituicao{Instituto Federal de Ciência e e e ee}



% O preambulo deve conter o tipo do trabalho, o objetivo, 
% o nome da instituição e a área de concentração 

\preambulo{Trabalho apresentado como requisito parcial para obtenção do título de
Licenciado em Computação  pelo Instituto Federal de Educação Ciência e Tecnologia da Bahia, \italico{campus} Santo Amaro, sob
orientação da Profa. \imprimirorientador . }

% Espaçamentos entre linhas e parágrafos 
% O tamanho do parágrafo é dado por:
\setlength{\parindent}{1.5cm}

% Controle do espaçamento entre um parágrafo e outro:
\setlength{\parskip}{0.2cm}  % tente também \onelineskip


%%%%%%%%%%% INDICE REMISSIVO

\informacoespdf

%%%%%%%%%%%% Início do Documento
\inicio


\pretextual
\capaifba

\folhaderostoifba

\folhadeaprovacaoifba


\imprimirdedicatoria{ a testes de ddfsfsdf dsf sdflksdfçksdmf  dkfjsdçfj kjsdf  sfçj }

\listoffigures*
\sumario
\textual


\nomefigura{dart.png}
\descricaoimagem{Imagem Dart}
\apelidoimagem{teste}
\legendaimagem{Criado por Everton}
\inserirfigura

Hoje

\nomefiguradois{dart.png}
\descricaoimagem{Nova imagem}
\apelidoimagem{Teste2}
\legendaimagem{Nova legenda}

\inserirfiguras
Agora


\apontarImagem{teste}

\descricaoimagemtres{Um}
\descricaoimagemdois{dois}
\legendaimagem{Nova fonte}

\apelidoimagemum{Figura1}

\subfiguras


\apontarImagem{Figura1}

\arquivo{bibliografia}

\referencias{REFERÊNCIAS}

\bibliografia



\fim



